\documentclass{article}
\usepackage[utf8]{inputenc}

\title{Homework 1}
\author{Benny Chen}
\date{\today}
 
\usepackage{color}
\usepackage{amsthm}
\usepackage{amssymb} 
\usepackage{amsmath}
\usepackage{listings}
\usepackage{xcolor}
\usepackage{listings}
\usepackage{graphicx}
\usepackage[hidelinks]{hyperref}
\usepackage{courier} 

\lstset{
tabsize = 4, %% set tab space width
showstringspaces = false, %% prevent space marking in strings, string is defined as the text that is generally printed directly to the console
numbers = left, %% display line numbers on the left
commentstyle = \color{green}, %% set comment color
keywordstyle = \color{blue}, %% set keyword color
stringstyle = \color{red}, %% set string color
rulecolor = \color{black}, %% set frame color to avoid being affected by text color
basicstyle = \small \ttfamily , %% set listing font and size
breaklines = true, %% enable line breaking
numberstyle = \tiny,
}

\begin{document}

\maketitle

\section{Problem A}

\subsection*{A.1}

\begin{lstlisting}[language = Java , frame = trBL , firstnumber = last , escapeinside={(*@}{@*)}]
//AreafromCircumference.java
//This program takes in the circumference(INPUT) and calculates the area of a circle(OUTPUT)

public class AreafromCircumference{
    public static void main(String[] args){
        //Set the value of pi to 22/7
        double pi = 22/7; 

        //Set the value of circumference to the input
        double circumference = Double.parseDouble(args[0]); 

        //Calculate the area of the circle
        double area = (circumference * circumference) / (4 * pi); 
        
        //Print the area of the circle
        System.out.printf("Area of circle is: %.2f\n", area); 
    }
}
\end{lstlisting}

\subsection*{Test Cases}

\begin{center}
    \includegraphics*[scale = .65]{./images/q1test.png}
\end{center}

\subsection*{A.2}

Can $\pi$ be encoded as “static final” in the code? Why or why not? Justify your answer.
\\
Yes, it can be encoded as "static final" in the code mainly due to the fact that we use $22/7$ as our approximation of $\pi$ and it will be a constant value that will not change throughout the program.
We would have to write is as 
\begin{lstlisting}[language = Java , frame = trBL , firstnumber = last , escapeinside={(*@}{@*)}]
static final double pi = 22.0 / 7.0; 
\end{lstlisting}

\section{Problem B}

\begin{lstlisting}[language = Java , frame = trBL , firstnumber = last , escapeinside={(*@}{@*)}]
// VendingChange.java
// This program takes in the amount of an item(INPUT) and calculates the change of $1 in the amount of quarters, dimes, and nickels(OUTPUT)

public class VendingChange {
    public static void main(String[] args){
        //Set the value of item to the input
        int item = Integer.parseInt(args[0]); 

        //Calculate the change
        int change = 100 - item; 

        //Calculate the amount of quarters
        int quarters = change / 25; 

        //Calculate the amount of dimes
        int dimes = (change % 25) / 10; 

        //Calculate the amount of nickels
        int nickels = ((change % 25) % 10) / 5; 

        //Print the change
        System.out.printf("You bought a item for %d cents and gave me a dollar, so your change is \n%d quarters, \n%d dimes, and \n%d nickels\n", item, quarters, dimes, nickels); 
    }       
}
    
\end{lstlisting}

\subsection*{Test Cases}

\begin{center}
    \includegraphics*[scale = .65]{./images/q2test.png}
\end{center}

\section{Problem C}

\begin{lstlisting}[language = Java , frame = trBL , firstnumber = last , escapeinside={(*@}{@*)}]
// OperatorPrecedence.java
// This program takes in 3 numbers(INPUT) and uses them in a equation to calculate the cube root of a number(OUTPUT)

public class OperatorPrecedence {
    public static void main(String[] args){
        //Set the value of x to the first input
        double x = Double.parseDouble(args[0]);

        //Set the value of y to the second input
        double y = Double.parseDouble(args[1]); 
        
        //Set the value of z to the third input
        double z = Double.parseDouble(args[2]); 
        
        //Plugs in values into equation then cubes it
        double answer = Math.cbrt(Math.pow(x, 2) + Math.pow(y, 2) - Math.abs(z)); 

        //Print the cube root
        System.out.printf("Cube Root is: %.2f\n", answer); 
    }
}
\end{lstlisting}

\subsection*{Test Cases}

\begin{center}
    \includegraphics*[scale = .65]{./images/q3test.png}
\end{center}

\section{Problem D}

\subsection*{D.1}

The problem the orginal code had was that only $m_2$ was being divided and not the whole $G * m_1 * m_2$ equation. By adding in parantheses, we can now divide the whole $G * m_1 * m_2$ equation by $r * r$.

\subsection*{D.2}

\begin{lstlisting}[language = Java , frame = trBL , firstnumber = last , escapeinside={(*@}{@*)}]
// Force.java
// This program takes in the mass of 2 objects(INPUT) and distance between centers of the masses(INPUT) and calculates the force(OUTPUT)

public class Force {
    public static void main(String[] args){
        //Set the value of m1 to the first input of mass
        double m1 = Double.parseDouble(args[0]); 

        //Set the value of m2 to the second input of mass
        double m2 = Double.parseDouble(args[1]); 

        //Set the value of r to the input of distance between centers of the masses
        double r = Double.parseDouble(args[2]); 

        //Calculate the force
        double force = (6.67 * Math.pow(10, -11)) * ((m1 * m2) / Math.pow(r, 2)); 

        //Print the force
        System.out.printf("Force is: %.2f\n", force); 
    }
}   
\end{lstlisting}

\subsection*{Test Cases}

\begin{center}
    \includegraphics*[scale = .65]{./images/q4test.png}
\end{center}

\end{document}